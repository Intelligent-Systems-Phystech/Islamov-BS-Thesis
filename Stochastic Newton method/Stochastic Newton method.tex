\documentclass[a4paper, 12pt]{article}
\usepackage{comment} % enables the use of multi-line comments (\ifx \fi) 
\usepackage{lipsum} %This package just generates Lorem Ipsum filler text. 
\usepackage{fullpage} % changes the margin
\usepackage[a4paper, total={7in, 10in}]{geometry}
\usepackage{amsmath}
\usepackage[utf8]{inputenc}
\usepackage[T2A]{fontenc}
\usepackage{amssymb}  
\usepackage{algorithm}
\usepackage{algorithmic}
\usepackage{graphicx}

%%%%%%%

%%% for coloring rows in a table
%\usepackage[flushleft]{threeparttable}
\usepackage{threeparttable}


\usepackage{multirow}
\usepackage{colortbl}
\definecolor{bgcolor}{rgb}{0.8,1,1}
\definecolor{bgcolor2}{rgb}{0.8,1,0.8}

%%%%%%%%
\usepackage{graphicx} %Loading the package
\graphicspath{{../Experiments/}}

\newcommand{\eqdef}{\; { := }\;}
\newcommand{\R}{\mathbb{R}}
\newcommand{\Exp}{\mathbb{E}}
\newcommand{\ExpBr}[1]{\mathbb{E}\left[#1\right]}
\newcommand{\norm}[1]{\left\|#1\right\|}
\def\<#1,#2>{\langle #1,#2\rangle}
\newcommand{\cX}{\mathcal{X}}
\newcommand{\bJ}{\mathbf{J}}
\newcommand{\dom}{\operatorname{dom}}
\newcommand{\prox}{\operatorname{prox}}
\newcommand{\footeq}[1]{{\footnotesize #1}}

\usepackage{tcolorbox}
\usepackage{pifont}
\definecolor{mydarkgreen}{RGB}{39,130,67}
\definecolor{mydarkred}{RGB}{192,47,25}
\newcommand{\green}{\color{mydarkgreen}}
\newcommand{\red}{\color{mydarkred}}
\newcommand{\cmark}{\green\ding{51}}%
\newcommand{\xmark}{\red\ding{55}}%

\newcommand{\mA}{\mathbf{A}}
\newcommand{\mB}{\mathbf{B}}
\newcommand{\mC}{\mathbf{C}}
\newcommand{\mH}{\mathbf{H}}
\newcommand{\mI}{\mathbf{I}}
\newcommand{\mU}{\mathbf{U}}
\newcommand{\mZ}{\mathbf{Z}}

\newcommand{\cC}{{\mathcal{C}}}
\newcommand{\cH}{{\mathcal{H}}}
\newcommand{\cW}{{\mathcal{W}}}

\usepackage{amsmath,amsfonts,amssymb,amsthm,array}


\usepackage{mdframed} 
\usepackage{thmtools}
\usepackage{textcomp}

%\theoremstyle{shaded}
\newcommand{\Prob}{\mathbb{P}}
\declaretheorem[within=section]{definition}
\declaretheorem[sibling=definition]{theorem}
\declaretheorem[sibling=definition]{proposition}
\declaretheorem[sibling=definition]{assumption}
\declaretheorem[sibling=definition]{corollary}
\declaretheorem[sibling=definition]{conjecture}
\declaretheorem[sibling=definition]{lemma}
\declaretheorem[sibling=definition]{example}
\declaretheorem[sibling=definition]{remark}

\newtheorem{def1}{Определение}
\newtheorem{lem1}{Лемма}
\newtheorem{theorem1}{Теорема}
%\newtheorem*{proof1}{\textit{Доказательство}}
\renewcommand{\proofname}{Доказательство}

\usepackage[russian]{babel}

% TO DO NOTES 
\usepackage[colorinlistoftodos,bordercolor=orange,backgroundcolor=orange!20,linecolor=orange,textsize=scriptsize]{todonotes}



\usepackage{microtype}
\usepackage{subfigure}
\usepackage{booktabs} % for professional tables


\usepackage{grffile}


\usepackage{hyperref}

\newcommand{\theHalgorithm}{\arabic{algorithm}}
\setlength{\marginparwidth }{2cm} 
\begin{document}
	\noindent{\bf Стохастический метод Ньютона}  
	\hfill\textbf{Исламов Рустем}
	\rule{7in}{1.8pt}\\
	
	\noindent Стандартная оптимизационная задача, которая возникает в машинном обучении имеет вид:
	\begin{equation}
		\min\limits_{x\in\R^d}\left[f(x) = \frac{1}{n}\sum\limits_{i=1}^n f_i(x)\right].
	\end{equation}

	\begin{def1}
		Дифференцируемая функция $\varphi:\R^d\rightarrow \R$ является сильно выпуклой функцией с константой $\mu$, если для любых $x, y\in \R^d$ выполнено
		\begin{equation}
		    \varphi(x) \geq \varphi(y) + \left<\nabla \varphi(y), x-y\right> +\frac{\mu}{2}\|x-y\|^2.
		\end{equation}
	\end{def1}
	\noindent Для дважды непрерывно дифференцируемой фукнции сильная выпуклость эквивалентна условию, что минимальное собственное значение гессиана ограничено снизу положительной константой, т.~е. $\nabla^2\varphi(x) \succeq \mu I,$ где $I \in \R^{d\times d}$~--- единичная матрица. Другими словами, $\lambda_{min}(\nabla^2\varphi(x)) \geq \mu.$
	
	\begin{def1}
		Дважды непрерывно дифференцируемая функция имеет липшецев гессиан с константой $H$, если для любых $x,y \in \R^d$ выполнено
	\begin{equation}
		\left\|\nabla^2 \varphi(x) - \nabla^2\varphi(y)\right\| \leq H\|x-y\|.
	\end{equation}
	\end{def1}
	\noindent Для дважды непрерывно дифференцируемой функции с липшецевым гессианом справедлива лемма
	\begin{lem1}
		Пусть функция $\varphi:\R^d \rightarrow \R$ дважды непрерывно дифференцируема, а ее гессиан липшецев с константой $H$. Тогда выполнены неравенства
		\begin{eqnarray}\label{lem1}
			\left\|\nabla \varphi(y) - \nabla\varphi(x) - \nabla^2 \varphi(x)(y-x)\right\|  \leq \frac{H}{2}\|y-x\|^2,\\
			\left|\varphi(y)-\varphi(x)-\left<\nabla \varphi(x),y-x\right> -\frac{1}{2}\left<\nabla^2\varphi(x)(y-x), y-x\right>\right| \leq \frac{H}{6}\|y-x\|^3.
		\end{eqnarray}
	\end{lem1}
	
	\begin{proof}[Доказательство]
		Действительно,
		\begin{eqnarray*}
			\left\|\nabla \varphi(y) - \nabla\varphi(x) - \nabla^2 \varphi(x)(y-x)\right\|	& = &\left\|\int\limits_0^1\left[\nabla^2 \varphi(x+\tau(y-x)) - \nabla^2 \varphi(x)\right](y-x)d\tau\right\|\\
			&\leq & H\|y-x\|^2\int\limits_0^1\tau d\tau = \frac{H}{2}\|y-x\|^2.
		\end{eqnarray*}	
		Следовательно, 
		\begin{eqnarray*}
			&&\left|\varphi(y)-\varphi(x)-\left<\nabla \varphi(x),y-x\right>-\frac{1}{2}\left<\nabla^2\varphi(x)(y-x), y-x\right>\right|\\
			&=& \left|\int\limits_0^1\left<\nabla \varphi(x+t(y-x))-\nabla\varphi(x)-t\nabla^2\varphi(x)(y-x),y-x\right>dt\right|\\
			&\leq&\frac{H}{2}\|y-x\|^3\int\limits_0^1t^2dt = \frac{H}{6}\|y-x\|^3.
		\end{eqnarray*}
	\end{proof}
	
	\section{Стохастический метод Ньютона}
	\noindent Будем аппроксимировать гессиан и градиент функции, используя последнюю доступную информацию, т.~е.
	\begin{equation*}
		\nabla^2 f(x^k) \approx \frac{1}{n}\sum\limits_{i=1}^n \nabla^2 f_i(w_i^k), \qquad \nabla f(x^k) \approx \frac{1}{n}\sum\limits_{i=1}^n\nabla f_i(w_i^k),
	\end{equation*}
	где $w_i^k$~--- последний вектор, для которого были посчитаны $\nabla f_i$ и $\nabla^2 f_i$.	
	
	
	\begin{algorithm}[h]
		\caption{{\bf Стохастический метод Ньютона}}
		\label{alg:SN}
		\begin{algorithmic}
			\STATE {\bfseries Initialize:} Choose starting iterates $w^0_1, w^0_2,\dots, w_0^n \in \R^d$ and minibatch size $\tau \in \{1, 2,\dots, n\}$
			\FOR{ $k = 0, 1, 2, \dots$}
				\STATE $x^{k+1} = \left[\frac{1}{n}\sum\limits_{i=1}^n \nabla^2 f_i(w_i^k)\right]^{-1}\left[\frac{1}{n}\sum\limits_{i=1}^n\nabla^2 f_i(w_i^k)w_i^k-\nabla f_i(w_i^k)\right]$
				
				\STATE Choose a subset $S^k \subseteq \{1,\dots, n\}$ of size $\tau$ uniformly at random
				
				\STATE $w_i^{k+1} = 
					\begin{cases}
						x^{k+1}, & i \in S^k\\
						w_i^k, & i \notin S^k
				 	\end{cases}$
			\ENDFOR
		\end{algorithmic}
	\end{algorithm} 
	
	\subsection{Локальная сходимость метода}
	\noindent Обозначим за $\Exp_k[\cdot] \eqdef \Exp[\cdot~|~x^k, w_1^k, \dots, w_n^k]$ условное матожидание, связанное со всей информацией предшествующей итерации $k+1$. Обозначим $x^*$ как решение исходной задачи. Введем функцию Ляпунова вида 
	$$\cW^k \eqdef \frac{1}{n}\sum\limits_{i=1}^n\|w_i^k-x^*\|^2.$$
	
	\begin{lem1}
	\label{lem2}
		Пусть функция $f_i$ является сильно выпуклой с константой $\mu$ и имеет липшецев гессиан с константой $H$ для всех $i \in \{1,2,\dots,n\}$. Тогда на следюущем шаге Алгоритма \ref{alg:SN} выполнено
		\begin{equation}
		\|x^{k+1}-x^*\|	\leq \frac{H}{2\mu}\cW^k.
		\end{equation}
	\end{lem1}
	\begin{proof}[Доказательство]
		Пусть $H^k \eqdef \frac{1}{n}\sum_{i=1}^n\nabla^2 f_i(w_i^k)$, тогда шаг алгоритма может быть записан в виде
		$$x^{k+1} = \left(H^k\right)^{-1}\left[\frac{1}{n}\sum\limits_{i=1}^n\nabla^2 f_i(w_i^k)w_i^k-\frac{1}{n}\sum\limits_{i=1}^n\nabla f_i(w_i^k)\right].$$
		Кроме этого, имеем $x^* = \left(H^k\right)^{-1}H^kx^*$ и $\sum_{i=1}^n\nabla f_i(x^*) = 0$. Тогда это ведет к равенству
		\begin{equation}
		\label{x_k_1_x_*} 
			x^{k+1} - x^* = \left(H^k\right)^{-1}\frac{1}{n}\sum\limits_{i=1}^n\left[\nabla^2f_i(w_i^k)(w_i^k-x^*)-\left(\nabla f_i(w_i^k)-\nabla f_i(x^*)\right)\right].
		\end{equation}
		Раз $f_i$ сильно выпукла, то $\nabla f_i^2(w_i^k) \succeq \mu I$ для всех $i$. Как следствие, $H^k \succeq \mu I$, что ведет к неравенству
		\begin{equation}
		\label{H_mu}
			\left\|\left(H^k\right)^{-1}\right\| \leq \frac{1}{\mu}.
		\end{equation}
		\begin{eqnarray*}
			\|x^{k+1}-x^*\| &\overset{\eqref{x_k_1_x_*}}{\leq}& \left\|\left(H^k\right)^{-1}\right\|\left\|\frac{1}{n}\sum\limits_{i=1}^n\left[\nabla^2 f_i(w_i^k)(w_i^k-x^*)-\left(\nabla f_i(w_i^k) - \nabla f_i(x^*)\right)\right]\right\|\\
			&\overset{\eqref{H_mu}}{\leq}&\frac{1}{\mu}\left\|\frac{1}{n}\sum\limits_{i=1}^n\left[\left(\nabla f_i(x^*)-\nabla f_i(w_i^k)\right)-\nabla^2 f_i(w_i^k)(x^*-w_i^k)\right]\right\|\\
			&\leq& \frac{1}{n\mu}\sum\limits_{i=1}^n\left\|\nabla f_i(x^*)-\nabla f_i(w_i^k)-\nabla^2 f_i(w_i^k)(x^*-w_i^k)\right\|\\
			&\overset{\eqref{lem1}}{\leq}& \frac{H}{2n\mu}\sum\limits_{i=1}^n\|w_i^k-x^*\|^2 = \frac{H}{2\mu}\cW^k.
		\end{eqnarray*}
	\end{proof}
	
	\begin{lem1}
	\label{lem3}
	Пусть каждая функция $f_i$ является сильно выпуклой с константой $\mu$ и имеет липшецев гессиан с константой $H$. Если $\|w_i^0-x^*\|\leq \frac{\mu}{H}$ для всех $i\in \{1,2,\dots,n\}$, тогда для всех $k$ имеем
	\begin{equation}
	\cW^k \leq \frac{\mu^2}{H^2}.
	\end{equation}
	\end{lem1}
	\begin{proof}[Доказательство]
	Покажем, что 
	\begin{equation}
	\label{claim}
		\|w_i^k-x^*\|^2 \leq \frac{\mu^2}{H^2}\quad \forall i \in \{1,2,\dots,n\}.
	\end{equation}
	Тогда ограничение сверху на $\cW^k$ будет следовать из этого.   Теперь докажем утверждение выше по индукции. Пусть оно верно для $k$, покажем, что оно верно и для $k+1$. Если $i\notin S^k$, то $w_i^k = w_i^{k+1}$, и утверждение выполнено по предположению индукции. Если $i \in S^k$, то 
	$$\|w_i^{k+1}-x^*\| = \|x^{k+1}-x^*\|\overset{\eqref{lem2}}{\leq} \frac{H}{2\mu}\frac{1}{n}\sum\limits_{j=1}^n\|w_j^k-x^*\|^2\leq \frac{H}{2\mu}\frac{1}{n}\sum\limits_{j=1}^n\frac{\mu^2}{H^2} < \frac{\mu}{H}.$$
	Поэтому мы снова получаем \eqref{claim} верно.
	\end{proof}
	
	\begin{lem1}
	\label{lem4}
		Случайные переменные Алгоритма \eqref{alg:SN} удовлетворяют равенству
		\begin{equation}
			\Exp_k\left[\cW^{k+1}\right] =\frac{\tau}{n}\Exp_k\left[\|x^{k+1}-x^*\|^2\right]+\left(1-\frac{\tau}{n}\right)\cW^k.
		\end{equation}
	\end{lem1}
	\begin{proof}[Доказательство]
		Для каждого $i$ переменная $w_i^{k+1}$ равна $x^{k+1}$ с вероятностью $\frac{\tau}{n}$ и равна $w_i^k$ с вероятностью $1-\frac{\tau}{n}$, что завершает доказательство.
	\end{proof}
	
	\begin{theorem1}
		Для случайных переменных Алгоритма \ref{alg:SN} верна рекурсия
		\begin{equation}
			\Exp_k\left[\cW^{k+1}\right] \leq \left(1-\frac{\tau}{n}+\frac{\tau}{n}\frac{H^2}{4\mu^2}cW^k\right)\cW^k.
		\end{equation}
		Более того, если $\|w_i^0-x^*\| < \frac{\mu}{H}$ для всех $i \in \{1,2,\dots,n\}$, тогда 
		\begin{equation}
			\Exp_k\left[\cW^{k+1}\right] \leq \left(1-\frac{3\tau}{4n}\right)\cW^k.
		\end{equation}
		Как следствие, если $\tau = n$, то Алгоритм \ref{alg:SN} имеет локальную квадратичную сходимость, как и стандартный метод Ньютона. Если $\tau=1$, то мы получаем локальную линейную сходимость, не зависящую от обусловленности функции, т.~е. метод адаптируется к кривизне функции.
	\end{theorem1}
	\begin{proof}[Доказательство]
		Используя Лемму \ref{lem2} и Лемму \ref{lem4}, имеем
		\begin{eqnarray*}
			\Exp_k\left[\cW^{k+1}\right]&\overset{\eqref{lem4}}{=}& \frac{\tau}{n}\|x^{k+1}-x^*\|^2+\left(1-\frac{\tau}{n}\right)\cW^k\\
			&\overset{\eqref{lem2}}{\leq}& \frac{\tau}{n}\frac{H^2}{4\mu^2}\left(\cW^k\right)^2+\left(1-\frac{\tau}{n}\right)\cW^k\\
			&=&\left(1-\frac{\tau}{n}+\frac{\tau}{n}\frac{H^2}{4\mu^2}\cW^k\right)\cW^k\\
			&\overset{\eqref{lem3}}{\leq}&\left(1-\frac{3\tau}{4n}\right)\cW^k,
		\end{eqnarray*}
		где на последнем шаге использовано предположение на ограниченность на $\|w_i^0-x^*\|$ для всех $i\in\{1,2,\dots,n\}$.
	\end{proof}

\end{document}
